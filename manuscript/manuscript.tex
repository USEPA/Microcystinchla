%%%%%%%%%%%%%%%%%%%%%%%%%%%%%%%%%%%%%%%%%%%%%%%%%%%%%%%%%%%%%%%%%%%%%%%%%%%%%
%% Original default rstudio/pandoc latex file
%% upated by @jhollist 09/15/2014
%% inspired by @cboetting https://github.com/cboettig/template and
%% @rmflight blog posts:
%% http://rmflight.github.io/posts/2014/07/analyses_as_packages.html 
%% http://rmflight.github.io/posts/2014/07/vignetteAnalysis.html).  
%%%%%%%%%%%%%%%%%%%%%%%%%%%%%%%%%%%%%%%%%%%%%%%%%%%%%%%%%%%%%%%%%%%%%%%%%%%%%

\documentclass[11pt,]{article}
\usepackage[T1]{fontenc}
\usepackage{lmodern}
\usepackage{amssymb,amsmath}
\usepackage{ifxetex,ifluatex}
\usepackage{fixltx2e} % provides \textsubscript
% use upquote if available, for straight quotes in verbatim environments
\IfFileExists{upquote.sty}{\usepackage{upquote}}{}
\ifnum 0\ifxetex 1\fi\ifluatex 1\fi=0 % if pdftex
  \usepackage[utf8]{inputenc}
\else % if luatex or xelatex
  \ifxetex
    \usepackage{mathspec}
    \usepackage{xltxtra,xunicode}
  \else
    \usepackage{fontspec}
  \fi
  \defaultfontfeatures{Mapping=tex-text,Scale=MatchLowercase}
  \newcommand{\euro}{€}
\fi
% use microtype if available
\IfFileExists{microtype.sty}{\usepackage{microtype}}{}
\usepackage{longtable,booktabs}
\usepackage{graphicx}
% Redefine \includegraphics so that, unless explicit options are
% given, the image width will not exceed the width of the page.
% Images get their normal width if they fit onto the page, but
% are scaled down if they would overflow the margins.
\makeatletter
\def\ScaleIfNeeded{%
  \ifdim\Gin@nat@width>\linewidth
    \linewidth
  \else
    \Gin@nat@width
  \fi
}
\makeatother
\let\Oldincludegraphics\includegraphics
{%
 \catcode`\@=11\relax%
 \gdef\includegraphics{\@ifnextchar[{\Oldincludegraphics}{\Oldincludegraphics[width=\ScaleIfNeeded]}}%
}%
\ifxetex
  \usepackage[setpagesize=false, % page size defined by xetex
              unicode=false, % unicode breaks when used with xetex
              xetex]{hyperref}
\else
  \usepackage[unicode=true]{hyperref}
\fi
\hypersetup{breaklinks=true,
            bookmarks=true,
            pdfauthor={},
            pdftitle={Associations between Chlorophyll a and various Microcystin-LR Health Advisory Concentrations},
            colorlinks=true,
            citecolor=blue,
            urlcolor=blue,
            linkcolor=magenta,
            pdfborder={0 0 0}}
\urlstyle{same}  % don't use monospace font for urls
\setlength{\parindent}{0pt}
\setlength{\parskip}{6pt plus 2pt minus 1pt}
\setlength{\emergencystretch}{3em}  % prevent overfull lines
\setcounter{secnumdepth}{5}

%%%%%%%%%%%%%%%%%%%%%%%%%%%%%%%%%%%%%%%%%%%%%%%%%%%%%%%%
%Changes borrowed from @cboettig, added by @jhollist 
% A modified page layout 
\textwidth 6.75in
\oddsidemargin -0.15in
\evensidemargin -0.15in
\textheight 9in
\topmargin -0.5in
\usepackage{lineno} % add 
%%%%%%%%%%%%%%%%%%%%%%%%%%%%%%%%%%%%%%%%%%%%%%%%%%%%%%%%

%%%%%%%%%%%%%%%%%%%%%%%%%%%%%%%%%%%%%%%%%%%%%%%%%%%%%%%%
%%Packages and layout changes by @jhollist 09/15/2014
\usepackage{ragged2e}
\usepackage[font=normalsize]{caption}
  \usepackage[doublespacing]{setspace}
\usepackage{parskip}
\usepackage{fancyhdr}
\pagestyle{fancy}
\fancyhf{}
\renewcommand{\headrulewidth}{0pt}
\rfoot{\today}
\lfoot{\thepage}
%%Changed default abstract width and added lines
\renewenvironment{abstract}{
  \hfill\begin{minipage}{1\textwidth}
  \rule{\textwidth}{1pt}\vspace{5pt}
  \normalsize
  \begin{justify}
  \bfseries\abstractname\vspace{5pt}
  \end{justify}}
  {\par\noindent\rule{\textwidth}{1pt}\end{minipage}
}
%%%%%%%%%%%%%%%%%%%%%%%%%%%%%%%%%%%%%%%%%%%%%%%%%%%%%%%%

\title{Associations between Chlorophyll \emph{a} and various Microcystin-LR
Health Advisory Concentrations}
\author{
Jeffrey W. Hollister
Betty J. Kreakie
}
\date{}

\begin{document}
%%Edited by @jhollist 09/15/2014
%%Adds title from YAML
\begin{singlespace}
\begin{center}
\huge Associations between Chlorophyll \emph{a} and various Microcystin-LR
Health Advisory Concentrations
\end{center}
%%Adds Author, correspond email asterisk, and affilnum from YAML
\begin{center}
\large
Jeffrey W. Hollister \textsuperscript{*} \textsuperscript{1} 
Betty J. Kreakie \textsuperscript{1} 
\end{center}
%%Adds affiliations from YAML
\begin{justify}
\footnotesize \emph{ 
\\*
\textsuperscript{1}US Environmental Protection Agency, Office of Research and Development,
National Health and Environmental Effects Research Laboratory, Atlantic
Ecology Division, 27 Tarzwell Drive Narragansett, RI, 02882, USA\\*
}
%%Adds corresponding author email(s) from YAML
\newcounter{num}
\setcounter{num}{1}
\\[0.1cm]
\footnotesize \emph{ 
\ifnum\value{num}=1%
\textsuperscript{*} corresponding author:
\fi
\href{mailto:hollister.jeff@epa.gov}{\nolinkurl{hollister.jeff@epa.gov}}
\stepcounter{num}
}
\end{justify}
%%Adds date from YAML
\normalsize

\end{singlespace}


\singlespace

\vspace{2mm}

\hrule

Cyanobacteria harmful algal blooms (cHABs) are associated with a wide
range of adverse health effects that stem mostly from the presence of
cyanotoxins. To help protect against these impacts, several health
advisory levels have been set for some toxins. In particular, one of the
more common toxins, microcystin, has several advisory levels set for
drinking water and recreational use and managing water bodies to meet
those levels could have far reaching benefits. However, compared to
other water quality measures, measurements of microcystin are not common
and current field measurement techniques have limited precision and
accuracy. Addressing these issues will take time and resources. Thus,
there is utility in finding indicators of microcystin that are already
widely available, can be estimated quickly and \emph{in situ}, and used
as a first defense against high levels of microcystin. In particular,
chlorophyll \emph{a} is very commonly measured, can be estimated
\emph{in situ}, and has been shown to be positively associated with
microcystin. In this paper we use this association to provide estimates
of chlorophyll \emph{a} that if exceeded would be indicative of a higher
probability of exceeding select health advisory concentrations for
microcystin-LR. Using the 2007 National Lakes Assessment and a
conditional probability approach that has been used in other water
quality settings, we identify chlorophyll \emph{a} concentrations that
are more likely than not to be associated with an exceedance of a
microcystin health advisory level. We look at the recent US EPA
standards for drinking water as well as the World Health Organization
levels for drinking water and recreational use. For microcystin
concentrations of 0.3, 1, 1.6, and 2 we find chlorophyll \emph{a}
concentrations of 23.68, 63.94, 89.71, and 102.6, respectively. When
managing for these various microcystin levels exceeding these reported
chlorophyll \emph{a} concentrations should be a trigger for further
testing and possible management action.

\vspace{3mm}

\hrule

\doublespace

\section{Introduction}\label{introduction}

In the summer of 2014, the city of Toledo, OH was forced to shut down
their municipal water supply due in part to an excess of microcystin-LR
that resulted from a ongoing cyanobacteria harmful algal bloom (cHAB) in
Lake Erie {[}REFS{]}. Since this event, significant legislation has been
passed in the United States and the US Environmental Protection Agency
(USEPA) has released suggested microcystin-LR concentrations that would
trigger health advisories. MORE ON THE LEVELS. While these levels and
associated advisories are likely to help mitigate the impacts from
harmful algal blooms, they are not without complications.

One of these complications is that they rely on available measurements
of microcystin-LR. This toxin can be measured in the field using test
strips but these are a coarse measure at best and currently available
test strips focus on 1 and 10 \(\mu\)g/L {[}REFS{]}. Measurements with
greater accuracy and precision require taking water samples and
processing those in a lab to determine the toxin concentration
{[}REFS{]}. Additionally, microcystin-LR is, currently, not a routinely
collected water quality parameter,thus, availability of microcystin-LR
data may limit our ability to screen water bodies for exceedances of the
various health advisories concentrations. Until microcystin-LR
concentrations are more widely collected an alternative measure is
needed. Fortunately, microcystin-LR has been shown to be associated with
several other, more easily measured components of water quality.

Chlorophyll \emph{a} is a very commonly measured components of water
quality that is also known to be associated with Microsystin-LR
concentrations {[}REFS{]}. Additionally there are many rapid
measurements for assessing chlorophyll \emph{a} levels \emph{in situ}.
For instance, there are small or hand held flourometers that provide
reliable measurements {[}REFS{]}. Given these facts, it might be
possible to identify chlorophyll \emph{a} concentrations that would be
associated with the various Microcystin-LR health advisory levels.
Identifying these associations would provide another reliable tool for
water resource managers to use to help manage the threat to public
health posed by cHABs and would be especially useful in the absence of
microcystin-LR concentrations. Thus, the goal of this paper is to utlize
the National Lakes Assessment data and identify chlorophyll \emph{a}
concentrations that are associated with higher probabilities of
exceeding several microcystin-LR health advisory concentrations {[}NLA
REF{]}. So that others may repeat or adjust this analysis, the data,
code, and this manuscript are freely available via
\href{https://github.com/USAPE/microcystinchla}{\url{https://github.com/USAPE/microcystinchla}}.

\section{Methods}\label{methods}

\subsection{Data}\label{data}

We used the 2007 National Lakes Assessment (NLA) water quality and
microcystin-LR concentration data {[}REF{]}. These data represent a
snapshot of water quality from the summer of 2007 and data on
chlorophyll \emph{a} and microcystin-LR concentrations are available for
lakes.

\subsection{Conditional Probability
Analysis}\label{conditional-probability-analysis}

We used a conditional probability analysis (CPA) approach to explore
associations between chlorophyll \emph{a} concentrations and World
Health Organization (WHO) and U.S. Environmental Protection Agency (U.S.
EPA) microcystin-LR health advisory levels (Paul and Munns 2011). Many
levels have been suggested (Table \ref{tab:microcystin_levels}), but
lakes with higher microcystin-LR concentrations in the NLA were rare.
Only 1.16 \% of lakes sampled had a concentration greater than 10. Thus,
for this analysis we focus on the microcystin concentrations that are
better represented in the NLA data. These were 0.3, 1, 1.6, and 2
\(\mu\)g/L.

A detailed discussion of CPA is beyond the scope of this paper, but see
Paul et al. {[}-REF{]} and Hollister et al. {[}-REF{]} for details. For
this analysis, we used CPA to examine how the conditional probability of
exceeding one of the health advisory changes as chlorophyll \emph{a}
increases in a lake. The 95\% confidence intervals were calculated from
1000 bootstrapped samples. To identify chlorophyll \emph{a}
concentrations of concern we used a 50\% conditional probability of
exceeding each health advisory level and extracted the minimum
chlorophyll \emph{a} concentration that was associated with the upper
confidence level being 50\% or greater. As both microcystin-LR and
chlorophyll \emph{a} values were both highely skewed right, a log base
10 transformation was used. Additional details of the specific
implementation are available at
\url{https://github.com/USEPA/microcystinchla}.

\section{Results}\label{results}

In the 2007 NLA, microcystin-LR concentrations ranged from 0.05 to 225.
Microcystin-LR concentrations of 0.05 \(\mu\)g/L represent the detection
limits. Any value greater than that indicates the presence of
microcystin-LR. Of those lakes with microcystin, the median
concentration was 0.51 and the mean was 0.51. Lastly, of all lakes
sampled, 21\% of lakes exceeded the U.S. EPA childrens drinking water
standard, 8.8\% of lakes exceeded the U.S. EPA adult drinking water
standard,11.7\% of lakes exceeded the WHO drinking water standard,and
7.3\% of lakes exceeded the WHO recreational standard. For chlorophyll
\emph{a}, the range was 0.07 to 936. All lakes had reported chlorophyll
\emph{a} concentrations that exceeded detection limits. The median
concentration was 7.79 and the mean was 29.6301946.

The association between Log 10 microcystin-LR and Log 10 chlorophyll
\emph{a} show a wedge pattern (Figure \ref{fig:chla_micro_scatter}).
This indicates that higher concentrations of microcystin-LR almost
always co-occur with higher concentrations of chlorophyll \emph{a} yet
the inverse is not true. Higher chlorophyll \emph{a} is not necessarily
predictive of higher microcystin-LR concentrations; however, chlorophyll
\emph{a} may be predictive of the probability of exceeding a certian
concentration. This is the case as the probability of exceeding each of
the four tested health adviroy levels shos this patern of probability of
exceeding the advisory level increases as a function of chlorophyll
\emph{a} concentration (Figure \ref{fig:multi_cp_plot}).

We use this association to identify chlorophyll \emph{a} concentrations
that are associated with greater than even odds of exceeding a given
health advisory level (Table \ref{tab:mc_chla_table}). These represent
27.8\%, 11.9\%, 8.3\%, and 7.4\% of sample lakes for the U.S EPA
childrens drinking water, the WHO drinking water, the U.S. EPA Adult
drinking water, and the WHO recreational standards, respecitvely.

\section{Discussion}\label{discussion}

\section{Figures}\label{figures}

\begin{figure}[htbp]
\centering
\includegraphics{manuscript_files/figure-latex/chla_micro_scatter-1.pdf}
\caption{Scatterplot showing association betweeen chlorophyll \textit{a}
and microcystin-LR. \label{fig:chla_micro_scatter}}
\end{figure}

\newpage

\begin{figure}[htbp]
\centering
\includegraphics{manuscript_files/figure-latex/epa_child_cp_plot-1.pdf}
\caption{Conditional probability plots showing association between the
probability of exceeding various microcystin-LR (MLR) health advisory
Levels. \label{fig:multi_cp_plot}}
\end{figure}

\newpage

\section{Tables}\label{tables}

\begin{longtable}[c]{@{}lll@{}}
\caption{Various suggested microcystin-LR health advisory
concentrations. \label{tab:microcystin_levels}}\tabularnewline
\toprule
Source & Type & Concentration\tabularnewline
\midrule
\endfirsthead
\toprule
Source & Type & Concentration\tabularnewline
\midrule
\endhead
WHO & Drinking & 1 \(\mu\)g/L\tabularnewline
U.S. EPA & Drinking & 0.3 \(\mu\)g/L\tabularnewline
U.S. EPA & Drinking & 1.6 \(\mu\)g/L\tabularnewline
WHO & Recreational & 2-4 \(\mu\)g/L\tabularnewline
WHO & Recreational & 10-20 \(\mu\)g/L\tabularnewline
WHO & Recreational & 20-2000 \(\mu\)g/L\tabularnewline
WHO & Recreational & \textgreater{}2000 \(\mu\)g/L\tabularnewline
\bottomrule
\end{longtable}

\newpage

\begin{longtable}[c]{@{}llrr@{}}
\caption{Chlorophyll \textit{a} concentrations that are associated with
a 50\% probability of exceeding a microcystin-LR health advisory
concentration. \label{tab:mc_chla_table}}\tabularnewline
\toprule
Source & Type & Microcystin & Chlorophyll\tabularnewline
\midrule
\endfirsthead
\toprule
Source & Type & Microcystin & Chlorophyll\tabularnewline
\midrule
\endhead
U.S. EPA & Drinking & 0.3 & 23.68\tabularnewline
WHO & Drinking & 1.0 & 63.94\tabularnewline
U.S. EPA & Drinking & 1.6 & 89.71\tabularnewline
WHO & Recreational & 2.0 & 102.60\tabularnewline
\bottomrule
\end{longtable}

\section*{References}\label{references}
\addcontentsline{toc}{section}{References}

Paul, J. F., and W. R. Munns. 2011. Probability surveys, conditional
probability, and ecological risk assessment. Environmental Toxicology
and Chemistry 30:1488--1495.

\end{document}
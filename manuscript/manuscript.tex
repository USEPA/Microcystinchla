%%%%%%%%%%%%%%%%%%%%%%%%%%%%%%%%%%%%%%%%%%%%%%%%%%%%%%%%%%%%%%%%%%%%%%%%%%%%%
%% Original default rstudio/pandoc latex file
%% upated by @jhollist 09/15/2014
%% inspired by @cboetting https://github.com/cboettig/template and
%% @rmflight blog posts:
%% http://rmflight.github.io/posts/2014/07/analyses_as_packages.html 
%% http://rmflight.github.io/posts/2014/07/vignetteAnalysis.html).  
%%%%%%%%%%%%%%%%%%%%%%%%%%%%%%%%%%%%%%%%%%%%%%%%%%%%%%%%%%%%%%%%%%%%%%%%%%%%%

\documentclass[11pt,]{article}
\usepackage[T1]{fontenc}
\usepackage{lmodern}
\usepackage{amssymb,amsmath}
\usepackage{ifxetex,ifluatex}
\usepackage{fixltx2e} % provides \textsubscript
% use upquote if available, for straight quotes in verbatim environments
\IfFileExists{upquote.sty}{\usepackage{upquote}}{}
\ifnum 0\ifxetex 1\fi\ifluatex 1\fi=0 % if pdftex
  \usepackage[utf8]{inputenc}
\else % if luatex or xelatex
  \ifxetex
    \usepackage{mathspec}
    \usepackage{xltxtra,xunicode}
  \else
    \usepackage{fontspec}
  \fi
  \defaultfontfeatures{Mapping=tex-text,Scale=MatchLowercase}
  \newcommand{\euro}{€}
\fi
% use microtype if available
\IfFileExists{microtype.sty}{\usepackage{microtype}}{}
\usepackage{longtable,booktabs}
\usepackage{graphicx}
% Redefine \includegraphics so that, unless explicit options are
% given, the image width will not exceed the width of the page.
% Images get their normal width if they fit onto the page, but
% are scaled down if they would overflow the margins.
\makeatletter
\def\ScaleIfNeeded{%
  \ifdim\Gin@nat@width>\linewidth
    \linewidth
  \else
    \Gin@nat@width
  \fi
}
\makeatother
\let\Oldincludegraphics\includegraphics
{%
 \catcode`\@=11\relax%
 \gdef\includegraphics{\@ifnextchar[{\Oldincludegraphics}{\Oldincludegraphics[width=\ScaleIfNeeded]}}%
}%
\ifxetex
  \usepackage[setpagesize=false, % page size defined by xetex
              unicode=false, % unicode breaks when used with xetex
              xetex]{hyperref}
\else
  \usepackage[unicode=true]{hyperref}
\fi
\hypersetup{breaklinks=true,
            bookmarks=true,
            pdfauthor={},
            pdftitle={Associations between Chlorophyll a and various Microcystin-LR Health Advisory Concentrations},
            colorlinks=true,
            citecolor=blue,
            urlcolor=blue,
            linkcolor=magenta,
            pdfborder={0 0 0}}
\urlstyle{same}  % don't use monospace font for urls
\setlength{\parindent}{0pt}
\setlength{\parskip}{6pt plus 2pt minus 1pt}
\setlength{\emergencystretch}{3em}  % prevent overfull lines
\setcounter{secnumdepth}{5}

%%%%%%%%%%%%%%%%%%%%%%%%%%%%%%%%%%%%%%%%%%%%%%%%%%%%%%%%
%Changes borrowed from @cboettig, added by @jhollist 
% A modified page layout 
\textwidth 6.75in
\oddsidemargin -0.15in
\evensidemargin -0.15in
\textheight 9in
\topmargin -0.5in
\usepackage{lineno} % add 
%%%%%%%%%%%%%%%%%%%%%%%%%%%%%%%%%%%%%%%%%%%%%%%%%%%%%%%%

%%%%%%%%%%%%%%%%%%%%%%%%%%%%%%%%%%%%%%%%%%%%%%%%%%%%%%%%
%%Packages and layout changes by @jhollist 09/15/2014
\usepackage{ragged2e}
\usepackage[font=normalsize]{caption}
  \usepackage[doublespacing]{setspace}
\usepackage{parskip}
\usepackage{fancyhdr}
\pagestyle{fancy}
\fancyhf{}
\renewcommand{\headrulewidth}{0pt}
\rfoot{\today}
\lfoot{\thepage}
%%Changed default abstract width and added lines
\renewenvironment{abstract}{
  \hfill\begin{minipage}{1\textwidth}
  \rule{\textwidth}{1pt}\vspace{5pt}
  \normalsize
  \begin{justify}
  \bfseries\abstractname\vspace{5pt}
  \end{justify}}
  {\par\noindent\rule{\textwidth}{1pt}\end{minipage}
}
%%%%%%%%%%%%%%%%%%%%%%%%%%%%%%%%%%%%%%%%%%%%%%%%%%%%%%%%

\title{Associations between Chlorophyll \emph{a} and various Microcystin-LR
Health Advisory Concentrations}
\author{
Jeffrey W. Hollister
Betty J. Kreakie
Dorothy Q. Kellog
}
\date{}

\begin{document}
%%Edited by @jhollist 09/15/2014
%%Adds title from YAML
\begin{singlespace}
\begin{center}
\huge Associations between Chlorophyll \emph{a} and various Microcystin-LR
Health Advisory Concentrations
\end{center}
%%Adds Author, correspond email asterisk, and affilnum from YAML
\begin{center}
\large
Jeffrey W. Hollister \textsuperscript{*} \textsuperscript{1} 
Betty J. Kreakie \textsuperscript{1} 
Dorothy Q. Kellog \textsuperscript{2} 
\end{center}
%%Adds affiliations from YAML
\begin{justify}
\footnotesize \emph{ 
\\*
\textsuperscript{1}US Environmental Protection Agency, Office of Research and Development,
National Health and Environmental Effects Research Laboratory, Atlantic
Ecology Division, 27 Tarzwell Drive Narragansett, RI, 02882, USA\\*
\\*
\textsuperscript{2}University of Rhode Island, Department of Natural Resrouces Science,
Kingston, RI, 02882, USA\\*
}
%%Adds corresponding author email(s) from YAML
\newcounter{num}
\setcounter{num}{1}
\\[0.1cm]
\footnotesize \emph{ 
\ifnum\value{num}=1%
\textsuperscript{*} corresponding author:
\fi
\href{mailto:hollister.jeff@epa.gov}{\nolinkurl{hollister.jeff@epa.gov}}
\stepcounter{num}
}
\end{justify}
%%Adds date from YAML
\normalsize

\end{singlespace}


\singlespace

\vspace{2mm}

\hrule

Cyanobacteria harmful algal blooms (cHABs) are associated with a wide
arrange of adverse health effects that stem mostly from the presence of
cyanotoxins. To help protect agains the impacts, several health advisory
levels have been set for some of these toxins, in particular, one of the
most common toxins, microcystin, has several advisory levels set for
drinking water and recreational use and managing water bodies to meet
those levels could have far reaching benefits. However, measuring
micorcystin can not currently be done \emph{in situ} and requires
samples be processed in a lab. This time consuming and expensive. It is
possible to find reliable indicators that may be estimated quickly and
\emph{in situ} as a first defense agains high level of microcystin. In
particular, chlorophyll \emph{a} has been shown to be postively
associated with microcystin. In this paper we use this assocation to
provide estimates of chlorophyll \emph{a} that if exceeded would be
indiciative of a higher likelihood of exceeding select concentrations of
microcystin. Using the 2007 National Lakes Assessment and a conditional
probability appoach that has been used in other water quality settings,
we idenfify chlorophyll \emph{a} concentrations that are more likely
than not to be associated with an exceedance of a microcystin health
advisory level. We look at the recent US EPA standards for drinking
water as well as the World Health Organization levels for drinking water
and recerational use. For microcystin concentrations of 0.3, 1, 1.6, 2.
and 4 we find chlorophyll \emph{a} concentrations of 23.68, 65.2, 79.8,
113.14, and 273.6, respectively. When managing for these various
microsystin levels exceed these reproted chlorophyll \emph{a}
concentratoins should be a trigger for further testing and possibly
managment action.

\vspace{3mm}

\hrule

\doublespace

\section{Introduction}\label{introduction}

In the summer of 2014, the city of Toledo, OH was forced to shut down
their municipal water supply due in part to an excess of Microcystin-LR
that resulted from a ongoing cyanobacterial harmful algal bloom (cHAB)
in Lake Erie. Since this event, signficant legislation has been passed
in the United States and the US Environmental Protection Agency (USEPA)
has released suggested microcystin-LR concentrations that would trigger
health advisories. While these levels and association advisories are
likey to help mitigate the impacts from harmful algal blooms, they are
not without complications.

One of these complications is that they rely on availble measurments of
Microcysin-LR which requires taking regular water samples and having
those samples process in a lab to determine the toxin concentration
{[}REFS{]}. This has the potentially to be costly and time consuming
both factors which could limit monitoring efforts. Fortunately,
microcystin-LR has been shown to be associated with several other, more
easily measured components of water quality.

Chlorophyll \emph{a} is one of the most commonly measured components of
water quality that is also known to be strongly associated with
Microsystin-LR concentrations {[}REFS{]}. Additionally there are many
rapid measurements for assessing chlorophyll \emph{a} levels \emph{in
situ}. For instance, there are small or hand held flourometers that
provide reliable measurements {[}REFS{]}. Given these facts, it might be
possible to identify chlorophyll \emph{a} concentrations that would be
associated with the various Microcystin-LR health advisory levels.
Identifying these associations would provide another reliable tool for
water resource managers to use to help manage the threat to public
health posed by cHABs.

Use association and cpa to id chl a concentration that indicative of
exceeding HA

\section{Methods}\label{methods}

\begin{longtable}[c]{@{}lll@{}}
\toprule
\begin{minipage}[b]{0.11\columnwidth}\raggedright\strut
Source
\strut\end{minipage} &
\begin{minipage}[b]{0.16\columnwidth}\raggedright\strut
Type
\strut\end{minipage} &
\begin{minipage}[b]{0.19\columnwidth}\raggedright\strut
Concentration
\strut\end{minipage}\tabularnewline
\midrule
\endhead
\begin{minipage}[t]{0.11\columnwidth}\raggedright\strut
WHO
\strut\end{minipage} &
\begin{minipage}[t]{0.16\columnwidth}\raggedright\strut
Drinking
\strut\end{minipage} &
\begin{minipage}[t]{0.19\columnwidth}\raggedright\strut
1 ug/l
\strut\end{minipage}\tabularnewline
\begin{minipage}[t]{0.11\columnwidth}\raggedright\strut
U.S. EPA
\strut\end{minipage} &
\begin{minipage}[t]{0.16\columnwidth}\raggedright\strut
Drinking
\strut\end{minipage} &
\begin{minipage}[t]{0.19\columnwidth}\raggedright\strut
0.3 ug/l
\strut\end{minipage}\tabularnewline
\begin{minipage}[t]{0.11\columnwidth}\raggedright\strut
U.S. EPA
\strut\end{minipage} &
\begin{minipage}[t]{0.16\columnwidth}\raggedright\strut
Drinking
\strut\end{minipage} &
\begin{minipage}[t]{0.19\columnwidth}\raggedright\strut
1.6 ug/l
\strut\end{minipage}\tabularnewline
\begin{minipage}[t]{0.11\columnwidth}\raggedright\strut
WHO
\strut\end{minipage} &
\begin{minipage}[t]{0.16\columnwidth}\raggedright\strut
Recreational
\strut\end{minipage} &
\begin{minipage}[t]{0.19\columnwidth}\raggedright\strut
2-4 ug/l
\strut\end{minipage}\tabularnewline
\begin{minipage}[t]{0.11\columnwidth}\raggedright\strut
WHO
\strut\end{minipage} &
\begin{minipage}[t]{0.16\columnwidth}\raggedright\strut
Recreational
\strut\end{minipage} &
\begin{minipage}[t]{0.19\columnwidth}\raggedright\strut
10-20 ug/l
\strut\end{minipage}\tabularnewline
\begin{minipage}[t]{0.11\columnwidth}\raggedright\strut
WHO
\strut\end{minipage} &
\begin{minipage}[t]{0.16\columnwidth}\raggedright\strut
Recreational
\strut\end{minipage} &
\begin{minipage}[t]{0.19\columnwidth}\raggedright\strut
20-2000 ug/l
\strut\end{minipage}\tabularnewline
\begin{minipage}[t]{0.11\columnwidth}\raggedright\strut
WHO
\strut\end{minipage} &
\begin{minipage}[t]{0.16\columnwidth}\raggedright\strut
Recreational
\strut\end{minipage} &
\begin{minipage}[t]{0.19\columnwidth}\raggedright\strut
\textgreater{}2000 ug/l
\strut\end{minipage}\tabularnewline
\bottomrule
\end{longtable}

We evaulated associated chlorohpyll \emph{a} concentrations for an
effect for each of the WHO and EPA levels. These were 0.3, 1, 1.6, 2, 4,
10, and 20 ug/l.

\subsection{Data and Study Area}\label{data-and-study-area}

\section{Results}\label{results}

\begin{longtable}[c]{@{}lll@{}}
\toprule
\begin{minipage}[b]{0.13\columnwidth}\raggedright\strut
Source
\strut\end{minipage} &
\begin{minipage}[b]{0.18\columnwidth}\raggedright\strut
Microcystin
\strut\end{minipage} &
\begin{minipage}[b]{0.18\columnwidth}\raggedright\strut
Chlorophyll
\strut\end{minipage}\tabularnewline
\midrule
\endhead
\begin{minipage}[t]{0.13\columnwidth}\raggedright\strut
EPA\_Child
\strut\end{minipage} &
\begin{minipage}[t]{0.18\columnwidth}\raggedright\strut
0.3
\strut\end{minipage} &
\begin{minipage}[t]{0.18\columnwidth}\raggedright\strut
23.68
\strut\end{minipage}\tabularnewline
\begin{minipage}[t]{0.13\columnwidth}\raggedright\strut
WHO
\strut\end{minipage} &
\begin{minipage}[t]{0.18\columnwidth}\raggedright\strut
1
\strut\end{minipage} &
\begin{minipage}[t]{0.18\columnwidth}\raggedright\strut
65.2
\strut\end{minipage}\tabularnewline
\begin{minipage}[t]{0.13\columnwidth}\raggedright\strut
EPA\_Adult
\strut\end{minipage} &
\begin{minipage}[t]{0.18\columnwidth}\raggedright\strut
1.6
\strut\end{minipage} &
\begin{minipage}[t]{0.18\columnwidth}\raggedright\strut
79.8
\strut\end{minipage}\tabularnewline
\begin{minipage}[t]{0.13\columnwidth}\raggedright\strut
WHO
\strut\end{minipage} &
\begin{minipage}[t]{0.18\columnwidth}\raggedright\strut
2
\strut\end{minipage} &
\begin{minipage}[t]{0.18\columnwidth}\raggedright\strut
113.1
\strut\end{minipage}\tabularnewline
\begin{minipage}[t]{0.13\columnwidth}\raggedright\strut
WHO
\strut\end{minipage} &
\begin{minipage}[t]{0.18\columnwidth}\raggedright\strut
4
\strut\end{minipage} &
\begin{minipage}[t]{0.18\columnwidth}\raggedright\strut
273.6
\strut\end{minipage}\tabularnewline
\begin{minipage}[t]{0.13\columnwidth}\raggedright\strut
WHO
\strut\end{minipage} &
\begin{minipage}[t]{0.18\columnwidth}\raggedright\strut
10
\strut\end{minipage} &
\begin{minipage}[t]{0.18\columnwidth}\raggedright\strut
338.4
\strut\end{minipage}\tabularnewline
\begin{minipage}[t]{0.13\columnwidth}\raggedright\strut
WHO
\strut\end{minipage} &
\begin{minipage}[t]{0.18\columnwidth}\raggedright\strut
20
\strut\end{minipage} &
\begin{minipage}[t]{0.18\columnwidth}\raggedright\strut
338.4
\strut\end{minipage}\tabularnewline
\bottomrule
\end{longtable}

\section{Discussion}\label{discussion}

\section{Figures}\label{figures}

\begin{figure}[htbp]
\centering
\includegraphics{manuscript_files/figure-latex/epa_child_cp_plot-1.pdf}
\caption{}
\end{figure}

\newpage

\begin{figure}[htbp]
\centering
\includegraphics{manuscript_files/figure-latex/who_drink_cp_plot-1.pdf}
\caption{}
\end{figure}

\newpage

\begin{figure}[htbp]
\centering
\includegraphics{manuscript_files/figure-latex/epa_adult_cp -1.pdf}
\caption{}
\end{figure}

\newpage

\begin{figure}[htbp]
\centering
\includegraphics{manuscript_files/figure-latex/who_rec_low1_cp-1.pdf}
\caption{}
\end{figure}

\newpage

\includegraphics{manuscript_files/figure-latex/who_rec_low2_cp-1.pdf}
\newpage

\includegraphics{manuscript_files/figure-latex/who_rec_med1_cp-1.pdf}
\newpage

\includegraphics{manuscript_files/figure-latex/who_rec_med2_cp-1.pdf}
\newpage

\section*{References}\label{references}
\addcontentsline{toc}{section}{References}

\end{document}
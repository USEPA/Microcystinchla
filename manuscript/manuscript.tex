%%%%%%%%%%%%%%%%%%%%%%%%%%%%%%%%%%%%%%%%%%%%%%%%%%%%%%%%%%%%%%%%%%%%%%%%%%%%%
%% Original default rstudio/pandoc latex file
%% upated by @jhollist 09/15/2014
%% inspired by @cboetting https://github.com/cboettig/template and
%% @rmflight blog posts:
%% http://rmflight.github.io/posts/2014/07/analyses_as_packages.html 
%% http://rmflight.github.io/posts/2014/07/vignetteAnalysis.html).  
%%%%%%%%%%%%%%%%%%%%%%%%%%%%%%%%%%%%%%%%%%%%%%%%%%%%%%%%%%%%%%%%%%%%%%%%%%%%%

\documentclass[11pt,]{article}
\usepackage[T1]{fontenc}
\usepackage{lmodern}
\usepackage{amssymb,amsmath}
\usepackage{ifxetex,ifluatex}
\usepackage{fixltx2e} % provides \textsubscript
% use upquote if available, for straight quotes in verbatim environments
\IfFileExists{upquote.sty}{\usepackage{upquote}}{}
\ifnum 0\ifxetex 1\fi\ifluatex 1\fi=0 % if pdftex
  \usepackage[utf8]{inputenc}
\else % if luatex or xelatex
  \ifxetex
    \usepackage{mathspec}
    \usepackage{xltxtra,xunicode}
  \else
    \usepackage{fontspec}
  \fi
  \defaultfontfeatures{Mapping=tex-text,Scale=MatchLowercase}
  \newcommand{\euro}{€}
\fi
% use microtype if available
\IfFileExists{microtype.sty}{\usepackage{microtype}}{}
\usepackage{longtable,booktabs}
\usepackage{graphicx}
% Redefine \includegraphics so that, unless explicit options are
% given, the image width will not exceed the width of the page.
% Images get their normal width if they fit onto the page, but
% are scaled down if they would overflow the margins.
\makeatletter
\def\ScaleIfNeeded{%
  \ifdim\Gin@nat@width>\linewidth
    \linewidth
  \else
    \Gin@nat@width
  \fi
}
\makeatother
\let\Oldincludegraphics\includegraphics
{%
 \catcode`\@=11\relax%
 \gdef\includegraphics{\@ifnextchar[{\Oldincludegraphics}{\Oldincludegraphics[width=\ScaleIfNeeded]}}%
}%
\ifxetex
  \usepackage[setpagesize=false, % page size defined by xetex
              unicode=false, % unicode breaks when used with xetex
              xetex]{hyperref}
\else
  \usepackage[unicode=true]{hyperref}
\fi
\hypersetup{breaklinks=true,
            bookmarks=true,
            pdfauthor={},
            pdftitle={Associations between Chlorophyll a and various Microcystin-LR Health Advisory Concentrations},
            colorlinks=true,
            citecolor=blue,
            urlcolor=blue,
            linkcolor=magenta,
            pdfborder={0 0 0}}
\urlstyle{same}  % don't use monospace font for urls
\setlength{\parindent}{0pt}
\setlength{\parskip}{6pt plus 2pt minus 1pt}
\setlength{\emergencystretch}{3em}  % prevent overfull lines
\setcounter{secnumdepth}{5}

%%%%%%%%%%%%%%%%%%%%%%%%%%%%%%%%%%%%%%%%%%%%%%%%%%%%%%%%
%Changes borrowed from @cboettig, added by @jhollist 
% A modified page layout 
\textwidth 6.75in
\oddsidemargin -0.15in
\evensidemargin -0.15in
\textheight 9in
\topmargin -0.5in
\usepackage{lineno} % add 
  \linenumbers % turns line numbering on 
%%%%%%%%%%%%%%%%%%%%%%%%%%%%%%%%%%%%%%%%%%%%%%%%%%%%%%%%

%%%%%%%%%%%%%%%%%%%%%%%%%%%%%%%%%%%%%%%%%%%%%%%%%%%%%%%%
%%Packages and layout changes by @jhollist 09/15/2014
\usepackage{ragged2e}
\usepackage[font=normalsize]{caption}
  \usepackage[doublespacing]{setspace}
\usepackage{parskip}
\usepackage{fancyhdr}
\pagestyle{fancy}
\fancyhf{}
\renewcommand{\headrulewidth}{0pt}
\rfoot{\today}
\lfoot{\thepage}
%%Changed default abstract width and added lines
\renewenvironment{abstract}{
  \hfill\begin{minipage}{1\textwidth}
  \rule{\textwidth}{1pt}\vspace{5pt}
  \normalsize
  \begin{justify}
  \bfseries\abstractname\vspace{5pt}
  \end{justify}}
  {\par\noindent\rule{\textwidth}{1pt}\end{minipage}
}
%%%%%%%%%%%%%%%%%%%%%%%%%%%%%%%%%%%%%%%%%%%%%%%%%%%%%%%%

\title{Associations between Chlorophyll \emph{a} and various Microcystin-LR
Health Advisory Concentrations}
\author{
Jeffrey W. Hollister
Betty J. Kreakie
}
\date{}

\begin{document}
%%Edited by @jhollist 09/15/2014
%%Adds title from YAML
\begin{singlespace}
\begin{center}
\huge Associations between Chlorophyll \emph{a} and various Microcystin-LR
Health Advisory Concentrations
\end{center}
%%Adds Author, correspond email asterisk, and affilnum from YAML
\begin{center}
\large
Jeffrey W. Hollister \textsuperscript{*} \textsuperscript{1} 
Betty J. Kreakie \textsuperscript{1} 
\end{center}
%%Adds affiliations from YAML
\begin{justify}
\footnotesize \emph{ 
\\*
\textsuperscript{1}US Environmental Protection Agency, Office of Research and Development,
National Health and Environmental Effects Research Laboratory, Atlantic
Ecology Division, 27 Tarzwell Drive Narragansett, RI, 02882, USA\\*
}
%%Adds corresponding author email(s) from YAML
\newcounter{num}
\setcounter{num}{1}
\\[0.1cm]
\footnotesize \emph{ 
\ifnum\value{num}=1%
\textsuperscript{*} corresponding author:
\fi
\href{mailto:hollister.jeff@epa.gov}{\nolinkurl{hollister.jeff@epa.gov}}
\stepcounter{num}
}
\end{justify}
%%Adds date from YAML
\normalsize

\end{singlespace}


\singlespace

\vspace{2mm}

\hrule

Cyanobacteria harmful algal blooms (cHABs) are associated with a wide
range of adverse health effects that stem mostly from the presence of
cyanotoxins. To help protect against these impacts, several health
advisory levels have been set for some toxins. In particular, one of the
more common toxins, microcystin, has several advisory levels set for
drinking water and recreational use. However, compared to other water
quality measures, field measurements of microcystin are not commonly
available due to cost and advanced understanding required to interpret
results. Addressing these issues will take time and resources. Thus,
there is utility in finding indicators of microcystin that are already
widely available, can be estimated quickly and \emph{in situ}, and used
as a first defense against high levels of microcystin. In particular,
chlorophyll \emph{a} is very commonly measured, can be estimated
\emph{in situ}, and has been shown to be positively associated with
microcystin. In this paper, we use this association to provide estimates
of chlorophyll \emph{a} concentrations that if exceeded would be
indicative of a higher probability of exceeding select health advisory
concentrations for microcystin-LR. Using the 2007 National Lakes
Assessment and a conditional probability approach, we identify
chlorophyll \emph{a} concentrations that are more likely than not to be
associated with an exceedance of a microcystin health advisory level. We
look at the recent US EPA standards for drinking water as well as the
World Health Organization levels for drinking water and recreational
use. For the specific advisory microcystin concentrations of 0.3, 1,
1.6, and 2, we find chlorophyll \emph{a} concentrations of 0.07, 0.07,
2.79, and 11.36, respectively. When managing for these various
microcystin levels, exceeding these reported chlorophyll \emph{a}
concentrations should be a trigger for further testing and possible
management action.

\vspace{3mm}

\hrule

\doublespace

\section{Introduction}\label{introduction}

Over the last decade or so numerous events and legislative activities
have raised the public awareness of harmful algal blooms (Rinta-Kanto et
al. 2009, HABHRCA 2014, Jetoo et al. 2015), and in response the US
Environmental Protection Agency (USEPA) has recently released suggested
microcystin-LR (one of the more common toxins) concentrations that would
trigger health advisories (McElhiney and Lawton 2005, Zurawell et al.
2005, USEPA 2015). Additionally, the World Health Organization (WHO) has
had proposed advisory levels for drinking water and a range of
recreational risk levels (Chorus and Bartram 1999, (WHO) and others
2003). While these levels and associated advisories are likely to help
mitigate the impacts from harmful algal blooms, they are not without
complications.

One of these complications is that they rely on available measurements
of microcystin-LR. While laboratory testing remains the gold standard
for quantifying microcystin-LR concentrations in water samples, several
field test kits have been developed. Even though field tests provide a
much needed means for rapid assessment, they are not yet widely used and
are moderately expensive (approximately \$150-\$200 depending on
specific kit) with a limited shelf life (typically one year) (James et
al. 2011, Aranda-Rodriguez et al. 2015). Additionally, each technique
requires nuanced understanding of the detection method (e.g., limit of
detection, specific microcystin variants being measured, and sampling
protocol).

Fortunately, microcystin-LR has been shown to be associated with several
other, more commonly measured and well understood components of water
quality that are readily assessed in the field. For instance, there are
small or hand held fluorometers that measure chlorohpyll \emph{a}.
Additionally, chlorophyll \emph{a} is a very commonly measured component
of water quality that is also known to be positively associated with
Microsystin-LR concentrations (Pip and Bowman 2014, Yuan et al. 2014).
Yuan et. al (2014) explore these associations in detail and control for
other related variables. In their analysis they find that total nitrogen
and chlorophyll \emph{a} show the strongest association with
microcystin. Furthermore, they identify chlorophyll\emph{a} and total
nitrogen concentrations that are associated with exceeding 1 \(\mu\)g/L
of microcystin. Given these facts, it should be possible to identify
chlorophyll \emph{a} concentrations that would be associated with the
new USEPA Microcystin-LR health advisory levels for drinking water.
Identifying these associations would provide another tool for water
resource managers to help manage the threat to public health posed by
cHABs and would be especially useful in the absence of microcystin-LR
concentrations.

In this paper we build on past efforts and utilize the National Lakes
Assessment (NLA) data and identify chlorophyll \emph{a} concentrations
that are associated with higher probabilities of exceeding several
microcystin-LR health advisory concentrations (Chorus and Bartram 1999,
USEPA 2009, 2015). We add to past studies by exploring associations with
newly announced advisory levels and by also applying a different method,
conditional probability analysis. Utilizing different methods
strengthens the evidence for suggested chlorophyll \emph{a} levels that
are associated with increased risk of exceeding the health advisory
levels as those levels are not predicated on a single analytical method.
So that others may repeat or adjust this analysis, the data, code, and
this manuscript are freely available via
\href{https://github.com/USAPE/microcystinchla}{\url{https://github.com/USAPE/microcystinchla}}.

\section{Methods}\label{methods}

\subsection{Data}\label{data}

We used the 2007 NLA water quality and microcystin-LR concentration data
(USEPA 2009). These data represent a snapshot of water quality from the
summer of 2007 for the conterminous United States. Data on chlorophyll
\emph{a} and microcystin-LR concentrations are available for lakes.

\subsection{Conditional Probability
Analysis}\label{conditional-probability-analysis}

We used a conditional probability analysis (CPA) approach to explore
associations between chlorophyll \emph{a} concentrations and World
Health Organization (WHO) and USEPA microcystin-LR health advisory
levels (Paul and Munns 2011). Many levels have been suggested (Table
\ref{tab:microcystin_levels}), but lakes with higher microcystin-LR
concentrations in the NLA were rare. Only 1.16 \% of lakes sampled had a
concentration greater than 10. Thus, for this analysis we focus on the
microcystin concentrations that are better represented in the NLA data.
These were the USEPA children's drinking water advisory level of 0.3
\(\mu\)g/L (USEPA Child), the WHO drinking water advisory level of 1
\(\mu\)g/L (WHO Drinking), the USEPA adult drinking water advisory level
of 1.6 \(\mu\)g/L (USEPA Adult), and the WHO recreational, low
probability of effect advisory level of 2 \(\mu\)g/L (WHO Recreational).

Conditional probability analysis provides information about the
probability of observing one event given another event has also occured.
For this analysis, we used CPA to examine how the conditional
probability of exceeding one of the health advisory changes as
chlorophyll \emph{a} increases in a lake. We expect to find higher
chlorohpyll \emph{a} concentrations to be associated with higher
probabilities of exceeding the microcystin-LR health advisory levels. We
also caclulated bootstrapped 95\% confidence intervals (CI) using 1000
bootstrapped samples. Thus, to identify chlorophyll \emph{a}
concentrations of concern we identfiy the value of the upper 95\% CI
across a range of conditional probabilities of exceeding each health
advisory level. As both microcystin-LR and chlorophyll \emph{a} values
were highely skewed right, a log base 10 transformation was used.
Additional details of the specific implementation are available at
\url{https://github.com/USEPA/microcystinchla}. A more detailed
discussion of CPA is beyond the scope of this paper, but see Paul et al.
(2005) and Hollister et al. (2008) for greater detail.

\section{Results}\label{results}

In the 2007 NLA, microcystin-LR concentrations ranged from 0.05 to 225
\(\mu\)g/L. Microcystin-LR concentrations of 0.05 \(\mu\)g/L represent
the detection limits. Any value greater than that indicates the presence
of microcystin-LR. Of those lakes with microcystin, the median
concentration was 0.51 and the mean was 3.17. Of all lakes sampled, 21\%
of lakes exceeded the USEPA Child level, 8.8\% of lakes exceeded the
USEPA Adult level,11.7\% of lakes exceeded the WHO Drinking level,and
7.3\% of lakes exceeded the WHO Recreational level. For chlorophyll
\emph{a}, the range was 0.07 to 936 \(\mu\)g/L. All lakes had reported
chlorophyll \emph{a} concentrations that exceeded detection limits. The
median concentration was 7.79 \(\mu\)g/L and the mean was 29.63
\(\mu\)g/L. The associations between chlorophyll \emph{a} and the upper
confidence interval across a range of conditional probability values is
shown in Table \ref{tab:mc_chla_table}. Specific chlorophyll \emph{a}
that are associated with greater than even odds of exceeding the
advisory levels were 0.07, 0.07, 2.79, and 11.36 for 0.3, 1.0, 1.6 and
2.0 \(\mu\)g/L advisory levels, respectively (Table
\ref{tab:mc_chla_table} \& Figure \ref{fig:multi_cp_plot}).

\section{Discussion}\label{discussion}

The association between Log10 microcystin-LR and Log10 chlorophyll
\emph{a} shows a wedge pattern (Figure \ref{fig:chla_micro_scatter}).
This indicates that higher concentrations of microcystin-LR almost
always co-occur with higher concentrations of chlorophyll \emph{a} yet
the inverse is not true. Higher chlorophyll \emph{a} is not necessarily
predictive of higher microcystin-LR concentrations; however, chlorophyll
\emph{a} may be predictive of the probability of exceeding a certain
concentration.

This is the case as the probability of exceeding each of the four tested
health advisory levels increases as a function of chlorophyll \emph{a}
concentration (Figure \ref{fig:multi_cp_plot}). We use this association
to identify chlorophyll \emph{a} concentrations that are associated with
greater than even odds of exceeding a given health advisory level (Table
\ref{tab:mc_chla_table}). These represent 99.9\%, 99.9\%, 77\%, and
43.8\% of sample lakes for the USEPA Child, the WHO Drinking, the USEPA
Adult, and the WHO recreational levels, respecitvely.

Furthermore, the chlorophyll \emph{a} cutoffs may be used to predict
whether or not a lake exceeds the microcystin-LR health advisories.
Doing so allows us to compare the accuracy of the prediction as well as
evaluate false negatives. Total accuracy of the four cutoffs predicting
microcystin-LR exceedances were 21\% for the USEPA children's advisory,
12\% for the WHO drinking water advisory, 31\% for the USEPA adult
advisory and, 61\% for the WHO regreational advisory (Tables
\ref{tab:child_conmat_table}, \ref{tab:who_drink_conmat_table},
\ref{tab:adult_conmat_table}, \& \ref{tab:who_rec_conmat_table}).
However, total accuracy is only one part of the prediction performace
with which we are concerned.

When using the chlorophyll \emph{a} cutoffs as an indicator of
microcystin-LR exceedances, the error that should be avoided is
predicting that no exceedance has occurred when in fact it has. In other
words, we would like to avoid Type II errors and minimize the proportion
of false negatives. For the four chlorophyll \emph{a} cut-offs we had a
proportion of false negatives of 0\%, 0\%, 0\% and , 1\% for the U.S EPA
childrens drinking water, the WHO drinking water, the USEPA adult
drinking water, and the WHO recreational advisories, respecitvely. In
each case we miss less than 10\% of the lakes that are in fact exceeding
the microcystin-LR advisory.

There are numerous possible uses for the chlorophyll \emph{a} and
microcystin-LR advisory cut-off values. First, in the absence of
microcystin-LR measurements, exceedence of the chlorophyll \emph{a}
concentrations could be a trigger for further actions. Given that there
is uncertainity around these chlorophyll \emph{a} cutoffs the best case
scenario would be to monitor for chlorophyll \emph{a} and in the event
of exceeding a target concentration take water samples and have those
samples tested in a lab for microcystin-LR.

A second potential use is to identify possible bloom events from
historical data. As harmful algal blooms are made up of many species and
have various mechanisms responsible for adverse impacts (e.g.~toxins,
hypoxia, odors), there is no single definition of a bloom. For cHABs one
approach has been to identify an increase over a baseline concentration
of phycocyanin (Miller et al. 2013). This is a useful approach for
targeted studies, but phycocyanin is also not always available and
measures the predominance of cyanobacterial pigments and not toxins.
Using our chlorophyll \emph{a} cutoffs provides a value that is more
directly associated with microcystin-LR and can be used to classify
lakes, from past surveys, as having bloomed.

Lastly, using chlorophyll \emph{a} is not meant as a replacement for
testing of microcystin-LR. It should be used when other, direct
measurements of cyanotoxins are not available. In those cases, which are
likely to be common at least in the near future, using a more ubiquitous
meausrement, such as chlorophyll \emph{a} will provide a reasonable
proxy for the probability of exceeding a microcystin-LR health advisory
level and provide better protection against adverse effects in both
drinking and recreational use cases.

\section{Acknowledgements}\label{acknowledgements}

\newpage

\section{Figures}\label{figures}

\begin{figure}[htbp]
\centering
\includegraphics{manuscript_files/figure-latex/chla_micro_scatter-1.pdf}
\caption{Scatterplot showing association betweeen chlorophyll \textit{a}
and microcystin-LR. \label{fig:chla_micro_scatter}}
\end{figure}

\newpage

\begin{verbatim}
## Loading required package: grid
\end{verbatim}

\begin{figure}[htbp]
\centering
\includegraphics{manuscript_files/figure-latex/epa_child_cp_plot-1.pdf}
\caption{Conditional probability plots showing association between the
probability of exceeding various microcystin-LR (MLR) health advisory
Levels. A.) Plot for USEPA Child (0.3 \(\mu\)g/L). B.) Plot for WHO
Drinking (1 \(\mu\)g/L). C.) Plot for USEPA Adult (1.6 \(\mu\)g/L). D.)
Plot for WHO Recreational (2 \(\mu\)g/L). \label{fig:multi_cp_plot}}
\end{figure}

\newpage

\section{Tables}\label{tables}

\begin{longtable}[c]{@{}lll@{}}
\caption{Various suggested microcystin-LR health advisory
concentrations. \label{tab:microcystin_levels}}\tabularnewline
\toprule
Source & Type & Concentration\tabularnewline
\midrule
\endfirsthead
\toprule
Source & Type & Concentration\tabularnewline
\midrule
\endhead
USEPA & Adult Drinking Water Advisory & 1.6 \(\mu\)g/L\tabularnewline
USEPA & Child Drinking Water Advisory & 0.3 \(\mu\)g/L\tabularnewline
WHO & Drinking Water & 1 \(\mu\)g/L\tabularnewline
WHO & Recreational: High Prob. of Effect & 20-2000
\(\mu\)g/L\tabularnewline
WHO & Recreational: Low Prob. of Effect & 2-4 \(\mu\)g/L\tabularnewline
WHO & Recreational: Moderate Prob. of Effect & 10-20
\(\mu\)g/L\tabularnewline
WHO & Recreational: Very High Prob. of Effect & \textgreater{}2000
\(\mu\)g/L\tabularnewline
\bottomrule
\end{longtable}

\newpage

\begin{longtable}[c]{@{}rrrrr@{}}
\caption{Chlorophyll \textit{a} concentrations that are associated with
a 50\% probability of exceeding a microcystin-LR health advisory
concentration. \label{tab:mc_chla_table}}\tabularnewline
\toprule
Cond. Probability & USEPA Child & WHO Drink & USEPA Adult & WHO
Recreational\tabularnewline
\midrule
\endfirsthead
\toprule
Cond. Probability & USEPA Child & WHO Drink & USEPA Adult & WHO
Recreational\tabularnewline
\midrule
\endhead
0.1 & 0.07 & 0.07 & 0.07 & 1.47\tabularnewline
0.2 & 0.07 & 4.42 & 11.47 & 19.04\tabularnewline
0.3 & 2.79 & 16.32 & 30.62 & 52.70\tabularnewline
0.4 & 11.36 & 38.30 & 69.19 & 82.22\tabularnewline
0.5 & 23.68 & 65.20 & 84.96 & 103.90\tabularnewline
0.6 & 38.30 & 98.46 & 136.66 & 125.40\tabularnewline
0.7 & 68.85 & 133.20 & 871.20 & 871.20\tabularnewline
0.8 & 114.62 & 338.40 & 871.20 & 871.20\tabularnewline
0.9 & 198.72 & 516.00 & 871.20 & 871.20\tabularnewline
\bottomrule
\end{longtable}

\newpage

\begin{longtable}[c]{@{}lrr@{}}
\caption{Confusion matrix comparing chlorophyll \textit{a} predicted
exceedences (rows) versus real exceedances (columns) for the USEPA
childrens drinking water advisory.
\label{tab:child_conmat_table}}\tabularnewline
\toprule
& FALSE & TRUE\tabularnewline
\midrule
\endfirsthead
\toprule
& FALSE & TRUE\tabularnewline
\midrule
\endhead
FALSE & 1 & 0\tabularnewline
TRUE & 810 & 217\tabularnewline
\bottomrule
\end{longtable}

\newpage

\begin{longtable}[c]{@{}lrr@{}}
\caption{Confusion matrix comparing chlorophyll \textit{a} predicted
exceedences (rows) versus real exceedances (columns) for the WHO
drinking water advisory.
\label{tab:who_drink_conmat_table}}\tabularnewline
\toprule
& FALSE & TRUE\tabularnewline
\midrule
\endfirsthead
\toprule
& FALSE & TRUE\tabularnewline
\midrule
\endhead
FALSE & 1 & 0\tabularnewline
TRUE & 906 & 121\tabularnewline
\bottomrule
\end{longtable}

\newpage

\begin{longtable}[c]{@{}lrr@{}}
\caption{Confusion matrix comparing chlorophyll \textit{a} predicted
exceedences (rows) versus real exceedances (columns) for the USEPA adult
drinking water advisory. \label{tab:adult_conmat_table}}\tabularnewline
\toprule
& FALSE & TRUE\tabularnewline
\midrule
\endfirsthead
\toprule
& FALSE & TRUE\tabularnewline
\midrule
\endhead
FALSE & 234 & 4\tabularnewline
TRUE & 703 & 87\tabularnewline
\bottomrule
\end{longtable}

\newpage

\begin{longtable}[c]{@{}lrr@{}}
\caption{Confusion matrix comparing chlorophyll \textit{a} predicted
exceedences (rows) versus real exceedances (columns) for the WHO
recreational water advisory.
\label{tab:who_rec_conmat_table}}\tabularnewline
\toprule
& FALSE & TRUE\tabularnewline
\midrule
\endfirsthead
\toprule
& FALSE & TRUE\tabularnewline
\midrule
\endhead
FALSE & 566 & 15\tabularnewline
TRUE & 387 & 60\tabularnewline
\bottomrule
\end{longtable}

\newpage

\section*{References}\label{references}
\addcontentsline{toc}{section}{References}

Aranda-Rodriguez, R., Z. Jin, J. Harvie, and A. Cabecinha. 2015.
Evaluation of three field test kits to detect microcystins from a public
health perspective. Harmful Algae 42:34--42.

Chorus, E. I., and J. Bartram. 1999. Toxic cyanobacteria in water: A
guide to their public health consequences, monitoring and management.

HABHRCA. 2014. Harmful algal bloom and hypoxia research and control
amendments act of 2014.

Hollister, J. W., H. A. Walker, and J. F. Paul. 2008. CProb: A
computational tool for conducting conditional probability analysis.
Journal of environmental quality 37:2392--2396.

James, R., A. Gregg, A. Dindal, and J. McKernan. 2011. Environmental
technology verification report: Abraxis microcystin test kits. Online
document. Accessed online: June 22.

Jetoo, S., V. I. Grover, and G. Krantzberg. 2015. The toledo drinking
water advisory: Suggested application of the water safety planning
approach. Sustainability 7:9787--9808.

McElhiney, J., and L. A. Lawton. 2005. Detection of the cyanobacterial
hepatotoxins microcystins. Toxicology and Applied Pharmacology
203:219--230.

Miller, T. R., L. Beversdorf, S. D. Chaston, and K. D. McMahon. 2013.
Spatiotemporal molecular analysis of cyanobacteria blooms reveals
microcystis-aphanizomenon interactions. PloS one 8:e74933.

Paul, J. F., and M. E. McDonald. 2005. Development of empirical,
geographically specific water quality criteria: A conditional
probability analysis approach 41:1211--1223.

Paul, J. F., and W. R. Munns. 2011. Probability surveys, conditional
probability, and ecological risk assessment. Environmental Toxicology
and Chemistry 30:1488--1495.

Pip, E., and L. Bowman. 2014. Microcystin and algal chlorophyll in
relation to nearshore nutrient concentrations in lake winnipeg, canada.
Environment and Pollution 3:p36.

Rinta-Kanto, J. M., E. A. Konopko, J. M. DeBruyn, R. A. Bourbonniere, G.
L. Boyer, and S. W. Wilhelm. 2009. Lake erie microcystis: Relationship
between microcystin production, dynamics of genotypes and environmental
parameters in a large lake. Harmful Algae 8:665--673.

USEPA. 2009. National lakes assessment: A collaborative survey of the
nation's lakes. ePA 841-r-09-001. Office of Water; Office of Research;
Development, US Environmental Protection Agency Washington, DC.

USEPA. 2015. Drinking water health advisory for the cyanobacterial
microcystin toxins. ePA-820R15100. Office of Water, US Environmental
Protection Agency Washington, DC.

(WHO), W. H. O., and others. 2003. Cyanobacterial toxins: Microcystin-lR
in drinking-water. background document for development of wHO guidelines
for drinking-water quality. geneva, switzerland. World Health
Organization, 2nd ed. Geneva.

Yuan, L. L., A. I. Pollard, S. Pather, J. L. Oliver, and L. D'Anglada.
2014. Managing microcystin: Identifying national-scale thresholds for
total nitrogen and chlorophyll a. Freshwater Biology 59:1970--1981.

Zurawell, R. W., H. Chen, J. M. Burke, and E. E. Prepas. 2005.
Hepatotoxic cyanobacteria: A review of the biological importance of
microcystins in freshwater environments. Journal of Toxicology and
Environmental Health, Part B 8:1--37.

\end{document}